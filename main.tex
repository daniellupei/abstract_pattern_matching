\documentclass{sig-alternate-05-2015}

\usepackage{mathtools,MnSymbol}
\usepackage{graphicx}
\usepackage{caption}
\usepackage{subcaption}
\usepackage{url}
%\usepackage{amsthm}


\begin{document}




\newcommand{\ident}[1]{\texttt{#1}}

\newcommand{\oR}{\overline{R}}
\newcommand{\qqquad}{\qquad\qquad}
\newcommand{\qqqquad}{\qqquad\qqquad}
\newcommand{\qqqqquad}{\qqqquad\qqqquad}





% Copyright
%\setcopyright{acmcopyright}
%\setcopyright{acmlicensed}
%\setcopyright{rightsretained}
%\setcopyright{usgov}
%\setcopyright{usgovmixed}
%\setcopyright{cagov}
%\setcopyright{cagovmixed}


\title{Jumping the ORDER BY barrier in large-scale pattern matching}

\begin{comment}
\numberofauthors{5} 
\author{
\alignauthor
Mike Barnett\\
   \affaddr{MSR, Redmond}\\
   \email{mbarnett@microsoft.com}
\alignauthor
Daniel Lupei\\
   \affaddr{EPFL}\\
   \email{daniel.lupei@epfl.ch}
\alignauthor
Saeed Maleki\\
   \affaddr{MSR, Redmond}\\
   \email{saemal@microsoft.com}
\and
\alignauthor
Madan Musuvathi\\
   \affaddr{MSR, Redmond}\\
   \email{madanm@microsoft.com}
\alignauthor 
Todd Mytkowicz\\
       \affaddr{MSR, Redmond}\\
       \email{toddm@microsoft.com}
}
\end{comment}

\maketitle
\begin{abstract}
Event-series pattern matching is a major component of large-scale data 
analytics pipelines enabling a wide range of system diagnostics tasks.
% (from tracing network bottlenecks to detecting security breaches). 
While existing solutions focus on the online scenario and expect 
their data to be (mostly) sorted on time, this is not the case in many 
situations where batch processing is preferable due to the large volume of data 
and where the physical layout has to meet wider system requirements. 
On map-reduce frameworks the consequence is an expensive ``shuffle the world'' 
stage wherein data are sorted and shuffled across the network.  
Moreover, existing systems treat the pattern matching engine as a black box 
which prevents them from optimizing the entire data analytics pipeline (and in 
particular, this costly shuffle).

We address these challenges two fold:
(i) we translate an expressive class of regular-expression like patterns to 
relational queries such that they can benefit from decades of progress in 
relational optimizers, and
(ii) we introduce the technique of {\em abstract pattern matching}, a linear 
time preprocessing step which, adapting ideas from symbolic execution and 
abstract interpretation, discards events from the input guaranteed not to 
appear in successful matches.  
Abstract pattern matching first computes in a conservative manner the 
output-relevant domain of every event variable in the pattern based on the 
(unary) predicates that reference it.
It then further refines these domains based on the structure of the pattern as 
well as its join predicates. 
The outcome is an {\em abstract filter} that when applied to the original 
stream excludes events that cannot form a complete match.

We implemented and applied {\em abstract pattern matching} in COSMOS/Scope to 
an industrial benchmark where we obtained up to 3 orders of magnitude reduction 
in shuffled data and 1.23x average speedup in total processing time.
% and 1.09x average speedup in end-to-end latency
\end{abstract}

\section{Introduction}

(Motivation) 
Event-series pattern matching is widely used for behavioral/anomaly/causality
analysis, for eg: network diagnostics, algorithmic
trading, security breach detection.


(Challenges)
Warehoused data is being analyzed by a wide range of queries thus it
may not be clustered/sorted on time. 

Analytics are being run both over fresh data as well as historical data,
therefore the data management system needs to support both online and
batching mode processing.


(Approach)
We design a set of relational-style optimizations for event-series pattern
matching workloads.

(Results)
We provide X reduction in shuffled data and Y reduction in end-to-end latency
for 3 workloads:
i) telemetry analysis over the events produced by the Windows event-reporting
infrastructure
ii) Github queries
iii) stock market algorithmic trading.



 



\section{Related Work}
\label{sec:rel_work}

Complex event processing has received extensive interest in the 
literature~\cite{SASE:2008,Akdere:2008,Brenna:2007,Cadonna:2011,Cadonna:2012,
	Demers:2007,Johnson:2007,Leghari:2015,Li:2007,Sadri:2004,SASE:2014,
	Raychev:2015} and 
has enjoyed similarly wide adoption in 
industry~\cite{esper_epl,oracle_mr,aster_npath,Carasso:2012},
with most of the work focused on online (near-real time) systems designed to 
deliver extremely low response times for mission-critical tasks (for eg., 
blocking a credit card upon detecting fraudulent activity). 
However, as the volume of data that needs to be processed continues to grow, 
online systems experience scalability issues considering that the 
pattern matching logic is typically implemented in a sequential manner.  


%(*) Distributed pattern matching

%(*) Pattern matching over out-of-order event series.


%(*) Stream processing engines optimizations

%(*) Pattern matching algorithms optimizations

%(*) Multiway join / semi-join optimizations

\section{Design}
\label{sec:design}

\begin{comment}   
Pattern matching algorithms require the input stream of events to be
(partially) sorted on time, while this is not always the case in practice due
to competing constraints of other stages in the data processing pipeline. 
\end{comment}
Evaluating pattern matching queries in a map-reduce framework usually 
adds a reduction step in order to sort the input, which can become the main 
bottleneck of the workload, both at the network level (large amounts of 
shuffled data) and at the processing level.
The standard approach to minimize the cost of sorting/data shuffling has been 
to introduce a {\em preprocessing} phase which first filters the input based on 
the {\em selection} predicates, i.e.\ removes all events that do not satisfy 
the guard of at least one event variable while ignoring its {\em join} 
predicates.
This {\em preprocessing} phase can significantly reduce both processing costs 
and latency since it takes linear time in the size of the input (as opposed to 
$O(nlogn)$ for sorting) and is embarrassingly parallel, i.e.\ scales out with 
the number of computing resources available.
Moreover, it can be merged with the previous operator in the data processing 
pipeline thus incurring no extra costs for materialization or data transfer.    



In our work we extend the opportunities for query plan optimizations across all 
the stages of the workload, beyond just pipelining the preprocessing phase of 
the pattern matcher, by leveraging the fact that a large class of patterns can 
be equivalently expressed as relational queries.
Whenever that is not the case we can narrow the scope of our optimizations to 
the sub-patterns that do.
The resulting relational expressions can then be optimized within the scope of 
the entire (predominantly relational) workload based on decades of progress in 
relational optimizations.  


Even in the scenarios where the relational optimizer decides that using the 
pattern matcher leads to the most efficient query plan, we can leverage the 
relational expressions to generate a {\em precise filter} which retains only 
those input events that appear within a complete match. 
It achieves that by fully exploiting the pattern's structure along with its 
{\em join} predicates, as opposed to just the {\em selection} predicates.   
Applying the precise filter as part of the preprocessing step leads to a 
dramatic improvement in its reduction ratio. 
This is unsurprising considering that the number of events forming complete 
matches is usually tiny compared to the input stream's size.
 

However in many cases the precise filter would be too expensive to build and 
evaluate as such.
Therefore, we coarsen it to obtain an {\em abstract} filter
which can be constructed and queried in a time and space efficient manner.
In particular, we make use of both {\em data} and {\em predicate} 
abstraction in order to generate a filter that, while conservative, closely 
matches the {\em precise} filter. 
Thus, in many cases we manage to discard most of the events that are guaranteed 
not to take part in a successful match and significantly reduce the amount of 
data fed into the pattern matcher.
We explore the trade-offs between the overheads incurred in building/querying
the filter and its accuracy. 


The derivation of abstract filters is not strictly tied to the ability to 
generate a semantically equivalent relational expression for a pattern. 
By adding a fixpoint construct, we demonstrate analogous techniques for 
generating both the {\em precise} and {\em abstract} filters.


\begin{comment}
We propose three levels of abstraction.
The first enforces the join constraints between different transitions as
expressed by join predicates within the transition guards.
The second one further imposes time windowing constraints (all events of a
successful match must occur within a timeout of the first event in the match).
Finally the last one enforces ordering constraints between {\em consecutive}
transitions of the pattern.
\end{comment}

\subsection{From patterns to relational queries}

To simplify the presentation we detail our approach on patterns specified as a 
finite automata where each transition is annotated by an event 
variable and a guard.
Turning regular expressions-like patterns to automata is straightforward,
and in many cases there is a direct mapping between the pattern's event 
variables and the automaton's transitions.
Figure~\ref{fig:serp_pattern} shows the automaton corresponding to the 
(S|E)R*P pattern, where S,R, and P, correspond to the same "Search", "Read 
review" and "Purchase" events described in section~\ref{sec:mot_example}, and E 
denotes the event of responding to a promotional email.
In the guards of the pattern, the $SE$ variable references either the $S$ or 
$E$ variable depending on the event that initiated the current match. 


\begin{figure}[t]
	\centering
	\includegraphics[clip, trim=0cm 10cm 0cm 3cm,width=\columnwidth]
	{graphs/example_sm3.pdf}
	\caption{Automata for the (S|E)R*P pattern.}
	\label{fig:serp_pattern}
\end{figure}


We formally define a finite automaton as $\cA = (S, T, s_{start}, C),$ 
where $S$ is the set of states, $T$ is the set of transitions, $s_{start}$ is 
the initial state and $C$ is the set of completion (accepting) states.
Each transition is defined in terms of the tuple $(X, p_X, src, dst)$ where $X$ 
is the variable binding the event currently considered by that transition, 
$p_X$ is the guard (a propositional formula) 
deciding whether the transition can be triggered or not, and $src$ and $dst$ 
are the transition's source and destination states.  
The atomic formulas of the guard are either {\em selection} predicates, which 
only reference the variable associated with the current transition,
or {\em join} predicates, which may also reference the variables of preceding 
transitions.
In particular, the guards of start transitions can only use selection 
predicates.  


We remark that the translation to relational expressions is not limited only to 
acyclic automata, but also to automata with cycles of fixed length, 
i.e.\ each iteration of the cycle has the same number of transitions.
This class of automata is of particular importance as it covers the 
vast majority of patterns found in benchmarks and in industrial workloads.  




{\bf Notation.}
We usually denote states by indices $i, j, k$, and we abuse notation to refer 
to transitions using the variable name they introduce (eg. $X, Y, Z$).
In addition we refer to states and transitions also as nodes and edges, 
respectively, in the corresponding graph of an automaton.

To streamline the presentation we begin by considering automata with only
cycles of length 1 and we distinguish between cycle transitions and
non-cycle transitions.


The translation process produces one relational query $Q_i$ per state $i$, and 
the final relational expression of the automata is obtained by unioning all the 
queries generated for the automata's accepting nodes.
Evaluating query $Q_i$ over a set of events computes partial matches, i.e.\ 
sequences of events, corresponding to all the possible paths between the 
starting node and node $i$.
Therefore, if $i$ is the starting node then its query returns an empty sequence 
while if $i$ is an accepting node then it returns complete matches found in the 
input stream of events.
Moreover, the partial (complete) matches computed by $Q_i$ do not include the 
events matched against cycle transitions therefore have a bounded length. 

The schema of the queries we generate consist of a sequence of event variables, 
one for each non-cycle transition that may occur along its associated set of 
paths. 
If a transition is triggered within a partial match, then its corresponding 
variable is initialized by the event that triggered it, otherwise that variable 
is assigned null.
For our example, the schema of query $Q_2$ consists of $S$, $E$ and $P$, and 
for each of its output tuples either $S$ or $E$ is set to null.


State queries $Q_i$ are defined as the union of transition queries $Q_X$ over 
all the incoming non-cycle transitions into state $i$, where each transition 
query $Q_X$ computes partial matches corresponding to the paths ending with 
transition $X$.
In turn, the $Q_X$ query corresponding to a non-starting, non-cycle transition 
is defined as the join between node query $Q_k$, where $k$ is the source of the 
transition, and the input relation of events, where each event considered is 
bound by variable $X$.
The condition enforced by $Q_X$ consists of the guard $p_X$ along with the 
constraint that the timestamp of $X$ succeeds the last event in the partial 
match produced by $Q_k$.
Additionally, a nested query ensures that no other events 
exist between the last event matched by $Q_k$ and the event bound by $X$, 
except for events matching cycle transitions starting and ending in $k$.
By contrast, for starting transitions we only need to apply the transition's 
guard $p_X$ over the input relation.

The translation process iterates in topological order over the nodes of the DAG
obtained by ignoring the cycle transitions of the automaton.
At each node $i$, it first generates the queries for all its incoming
non-cycle transitions $Q_X$ and then $Q_i$ is defined as their union. 
The schema of $Q_i$ is established as the union of the schemas of the incoming 
transitions $Q_X$.

Applying the procedure outlined above to our example produces the following 
queries:
\begin{align*}
%Q_0 =& \{\tuple{}\}
%\quad
Q_S\!=& \{ S \mid S \in \Ev : p_S \}
\;\;
Q_E\!= \{ E \mid E \in \Ev : p_E \}
\;\;
Q_1\!= Q_S \cup Q_E
\\
Q_P\!=& \{ \tuple{SE,P} \mid SE \in Q_1, P \in \Ev:
p_P\ .\ (\facc{SE}{t} < \facc{P}{t}) . 
\\
& 
\qqquad\quad
\{ R \mid R \in \Ev : \facc{R}{t} \in (\facc{SE}{t}, \facc{P}{t})\ .\ 
						! p_R \} = \emptyset
 \}
\\
Q_2\!=& Q_P
\end{align*}

{\bf Translating multi-transition cycles} of fixed length 
($\geq 1$) to relational queries
requires that we first normalize them such that each cycle
has a single starting node and a single ending node, and that the two coincide.
A starting node for a cycle is defined as the destination of one of its 
incoming edges, while an ending node is the source of one of its outgoing edges.
We remark, that cycles of fixed length cannot have transversal edges (paths), 
i.e.\ edges (paths) that connect non-adjacent nodes in the cycle, as this would 
violate the restriction that each of the cycle's iterations has the same length.


Given an automaton with a cycle that has multiple starting and ending nodes 
we first duplicate the cycle for each additional starting node. 
Then, for the resulting cycles we duplicate the path between their starting 
node and their last ending node (i.e.\ the furthest from the starting node).  
Finally, we change the source of each outgoing edge to the corresponding 
node in the newly created path, resulting in a cycle whose incoming and 
outgoing edges have the same node as destination and source, respectively.
By applying this procedure to every cycle with multiple starting and ending 
nodes we obtain a normalized automaton.  
We remark that, while the resulting automaton may have multiple cycles starting 
and ending with the same node, all of those must also have the same length.


The only part of the translation process that changes when generalizing from 
automata with single edge cycles to normalized automata is the specification of 
non-cycle transition queries $Q_X$, and in particular, the specification of its 
nested query should the source state $k$ of $X$ be the starting/ending point of 
a cycle.
We recall that in the case of cycles with a single transition $Y$ the nested 
query enforces that all events occurring in the interval between the last event 
in the partial match computed by $Q_k$ and the timestamp of event variable $X$ 
satisfy guard $p_Y$. 
By contrast, in the case of multi-transition fixed length cycles, for each 
event in the same interval we establish its position (based on  the count of 
events with smaller timestamps) and we ask that it satisfies the guard of the 
transition corresponding to that position in the cycle modulo the length of the 
cycle.
If multiple cycles initiate and conclude at the same node, we alternatively 
have to enforce that all events in an iteration satisfy the corresponding 
transition guards of a particular cycle.


\subsection{Precise filter generation}
\label{sec:prec_filter_generation}



After translating patterns into relational queries a host of relational 
optimizations become applicable, from column pruning and partial aggregation to
the selection of specific join algorithms.
In the following we detail our proposal for speeding up pattern matching in a 
distributed environment based on its representation in the relational world as 
a series of unions and joins.
The first step in this process is to derive a {\em precise} filter which 
retains from the input relation only those events guaranteed to appear in a 
successful match.
While it is understood that constructing and applying such filters may prove 
too expensive to evaluate directly, we discuss them nonetheless as they are 
essential in guiding the design of the {\em abstract} filter, its time and 
space efficient variant. 



% Even though this representation may prove too expensive to evaluate directly, 
% we use it as a stepping stone for building filters designed to discard from 
% the input (almost) all the events that do not participate in a complete match 
% and thus vastly reduce the number of events ultimately processed by the 
% pattern matching engine.
 
% discuss difference between fields referenced by selection predicates and 
% fields referenced by join predicates, and how the symbolic filters are 
% concerned with the latter.






The precise filter of an automaton $\cA$ consists of multiple components, one 
for each of its transitions, and an input event is rejected if it does not 
satisfy any of these components. 
In current work we derive precise filters only for non-cycle transitions and 
single-edge cycles, since the filters for transitions in multi-edge cycles 
are impractical to build/apply and abstract over, as they require the position 
of the considered event within a particular time interval.
 

The precise filter $\precs{X}$ corresponding to a non-cycle transition $X$ of 
automaton $\cA$ is extensionally defined in terms of the events 
from the input that bind the event variable $X$ in the output of its 
semantically equivalent relational query $Q_{\cA}$. 
Therefore $\precs{X}$ can be obtained from the definition of $Q_{\cA}$ by 
projecting away (i.e.\ existentially quantifying) all the other event variables 
in its output besides $X$. 
In our running example the precise filter derived for transition $P$ is:
\begin{align*}
&\precs{P}(P) \equiv \exists SE \in Q_1:
p_P\ .\ (\facc{SE}{t} < \facc{P}{t})\ . 
\\ 
&\qqquad\quad
\{ R \mid R \in \Ev : \facc{R}{t} \in (\facc{SE}{t}, \facc{P}{t})\ .\ 
! p_R \} = \emptyset
\end{align*}

The precise filter $\precs{Y}$ of a cycle transition $Y$, with node $k$ as 
source 
and destination, selects from the input those events that occur within interval 
$(t_Z, t_W)$, where $t_Z, t_W$, are the timestamps of a pair of event variables 
$Z, W$, from the output of $Q_{\cA}$ such that $Z, W$ are associated to 
non-cycle transitions entering and respectively exiting $k$.
We note that $\precs{Y}$ does not need to enforce the guard $p_Y$ as it is 
guaranteed
that all events between $t_Z$ and $t_W$ satisfy $p_Y$ based on the
nested query generated as part of the definition of $Q_W$ (and which was found 
to hold during the evaluation of $Q_{\cA}$).
For the cycle transition $R$ in our example we generate the following filter:
\begin{align*}
\precs{R}(R) \equiv \exists \tuple{SE,P} \in Q_{\cA}: 
 \facc{R}{t} \in (\facc{SE}{t}, \facc{P}{t})
\end{align*}
   



\begin{comment} 
The two kinds of transitions (cycle vs non-cycle) generate distinctly different 
kinds of symbolic filters. 
While both make use of the relational query $Q_{\cA}$ generated for the 
automaton, only the filter for cycle transitions makes direct use of its results
while the other simply uses $Q_{\cA}$'s expression as a starting point for its 
definition.
This distinction plays an important role in how we go about building these 
filters in a distributed environment.
\end{comment}


We take a bottom-up approach to building the symbolic filters as it allows us 
to outline an evaluation strategy that operates over sets and which uses set 
operations like union, intersection, set membership or emptiness testing. 
Adopting such a set-centric evaluation strategy is advantageous in a 
distributed environment due to the embarrassingly parallel nature of many set 
operators, but more importantly it gives us a powerful knob in terms of the set 
representations that we use, making it possible to trade off precision in favor 
of performance. 
This strategy is what ultimately guides the design of {\em abstract} 
filters (discussed in sections~\ref{sec:data_abstraction} 
and~\ref{sec:pred_abstraction}), which make our solution practical.
   

As a first step we build a {\em symbolic set} $\syms{X}$ for every transition 
variable $X$ of an automaton $\cA$, 
which collects the values of $X$'s fields joined throughout all the transition 
guards of $\cA$, as $X$ is bound to the input events that satisfy $p_X$'s
selection predicates.
We then re-write the precise filters by replacing each event variable and 
selection predicate associated to a transition with its corresponding symbolic 
set.
Finally, the join predicates get re-written in terms of slicing, intersection 
and emptiness testing over these sets.

The process is showcased in section~\ref{sec:mot_example} where we derive 
symbolic sets $\syms{S}, \syms{R}$ and $\syms{P}$, which we then use in the 
definition of precise filters $\precs{S}, \precs{R}$ and $\precs{P}$ along with 
slicing operators that express the join predicates of the pattern.
We omit the minute technical details of turning transition guards into 
expressions over symbolic sets as they are not particularly challenging.
It involves turning the guards into disjunctive normal form, and replacing in 
each conjunct the transition variables and their selection predicates with the 
corresponding symbolic set, and finally using slicing to encode their join 
predicates.    

\subsection{Data abstraction}
\label{sec:data_abstraction}


Building the precise filters as described in the previous section is not a 
feasible option, as it would require at least just as much work as computing 
the final result (for eg., for $\precs{R}$).
Nonetheless, the precise filters provide a template for designing 
appropriate set abstractions that support the set operations required for 
constructing and querying them in a time and space efficient manner. 
Our approach is to use set abstractions that sacrifice precision, while 
remaining conservative, as we should never eliminate events that would 
otherwise have contributed to a successful match.
We call the resulting structures {\em abstract} filters as they are the outcome 
of employing several abstractions during the process outlined for deriving the 
precise filters.  


First of all, we remark that the sets we abstract over contain tuples as 
opposed to single values, where each tuple field holds the values relevant to a 
particular join predicate. 
Similarly, the abstractions we chose need to be multidimensional in the sense 
that they must allow the testing of the domain of values corresponding to a 
specific join predicate, independent of the others.  
Second, we note that besides supporting set intersection and set union, the 
choice for a particular set abstraction is deeply influenced by the particular 
kind of predicates said abstractions needs to support.
For example, in the case of equality joins an appropriate data abstraction 
would be to use Bloom filters as they provide a low cost solution for testing 
whether o value belongs to a set, with the guarantee of no false negatives.
For enforcing inequality joins on the other hand, an interval map, i.e.\ a 
bit vector where each bit stands for a particular interval in the domain,
provides a similarly low cost abstraction.  

Finally, depending on whether the symbolic sets are tested for non-emptiness or 
emptiness, their abstraction has to provide either an over- or an 
under-approximation of the original contents.
For instance, when abstracting over the precise filter $\precs{S}$ from the 
example in section~\ref{sec:mot_example}, we have to use over-approximating set 
abstractions for $\syms{S}$, while the opposite is true for $\syms{\oR}$.
While this is of no concern for set abstractions, like interval maps, which 
support both, it does pose a challenge to abstractions based on hashing, like 
Bloom filters, which can only provide over-approximations.


One way to address the issue raised by hash-based data abstraction wrt.\ sets 
that require under-approximation is to replace them with the coarsest 
under-approximation possible, i.e. the empty set $\emptyset$ (as showcased in 
section~\ref{sec:mot_example}).
Alternatively, one can re-write the precise filter using min aggregates:
\begin{align*}
&
\PrecedesPP(\symv{s}) \equiv 
\min \{ 
\facc{\symv{p}}{t} \mid 
\tuple{\facc{\symv{p}}{t}, \_} \in 
\slice{\syms{P}}
{(\facc{\symv{s}}{t}\,:\,\facc{\symv{s}}{t} + t_{out}),\; 
	\facc{\symv{s}}{user}}
\}
\\
&\qquad
< \min \{ 
\facc{\symv{r}}{t} \mid 
\tuple{\facc{\symv{r}}{t}, \_} \in 
\slice{\syms{\oR}}
{(\facc{\symv{s}}{t}\,:\,\facc{\symv{s}}{t} + t_{out}),\; 
	\facc{\symv{s}}{user}} 
\},
\end{align*}
which captures the fact that a Search event is part of a complete match if the 
next Purchase event precedes the next event different from Read-review.
Now, we can safely use hashing to abstract over user ids as follows:
\begin{align*}
&
\PrecedesPP^{\hashid{user}}(\symv{s}) \equiv 
\\
&\qquad
\min_{u \in \hashid{\facc{s}{user}}}
\min \{ 
\facc{\symv{p}}{t} \mid 
\tuple{\facc{\symv{p}}{t}, \_} \in 
\slice{\syms{P}}
{(\facc{\symv{s}}{t}\,:\,\facc{\symv{s}}{t} + t_{out}),\; 
	u}
\}
\\
&\qquad
< 
\max_{u \in \hashid{\facc{s}{user}}}
\min \{ 
\facc{\symv{r}}{t} \mid 
\tuple{\facc{\symv{r}}{t}, \_} \in 
\slice{\syms{\oR}}
{(\facc{\symv{s}}{t}\,:\,\facc{\symv{s}}{t} + t_{out}),\; 
	u} 
\}.
\end{align*}

While for this example it was possible to come up with an alternative 
formulation of the precise filter, this may not always be the case. 
And even if it is possible, the alternatives may be too expensive to
materialize and query (for eg.\ evaluating $\PrecedesPP^{\hashid{user}}$ can 
easily be more costly than $\PrecedesP^{\interval{t}}$).
Therefore, in the next section we discuss how {\em predicate abstraction} can 
mitigate these kinds of issues.

\subsection{Predicate abstraction}
\label{sec:pred_abstraction}


We propose {\em predicate abstraction}, ie.\ the technique of weakening
 the
precise filters by discarding some of their predicates, as a way of
 overcoming
the challenges that can arise when turning them into abstract 
filters.
For example, it may happen that for some predicate types
(for eg.\ $x.Contains(y)$, where $x$, $y$ are strings) we simply cannot
 provide
any data abstraction, and even for those that we can, materializing and
 querying
those data abstractions might prove too expensive.




Predicate abstraction is an essential component of our approach
allowing us to strike the right balance between the data reduction that the
abstract filters provide on one hand, and the overheads introduced by their
data abstractions on the other.
For instance, we may choose to discard
predicates that have very low selectivity, i.e.\ the reduction in input 
data
that they provide does not justify the cost of enforcing them.
Similarly, one may turn to predicate abstraction when dealing with patterns
 with
a large number of join predicates or transitions, in order to mitigate the
increased overheads incurred by their data abstractions.


Given a join predicate $\theta(X.f, Y.f)$, the definition of precise filter 
$\precs{X}(\symv{x})$ is bound to contain a corresponding slicing of $Y$'s 
symbolic set as $\slice{\syms{Y}}{...,f}$. 
Just like in the case of data abstraction, depending on whether predicate 
$\theta$ appears in a negated sub-clause or not, its abstraction needs to be 
under or over approximating, in order to preserve conservativeness.
More precisely, we enforce the over-approximation of $\theta$ by assuming that 
the $f$ dimension of $\syms{Y}$ is invariably the entire domain of $f$.
Analogously, the under-approximation of $\theta$ leads to the assumption that 
the $f$ dimension of $\syms{Y}$ is invariably void, which by consequence 
reduces the entire $\syms{Y}$ to the empty set.
In section~\ref{sec:mot_example}, it was the under-approximating abstraction of 
predicate $\facc{S}{user} = \facc{R}{user}$ that produced the relaxed filter
$\PrecedesPPP$, by turning $\syms{\oR}$ in $\PrecedesP$ to the empty set.  
 
Finally, we remark that predicate abstraction need not be applied symmetrically,
i.e.\ given join predicate $\theta(X.f, Y.f)$ one could choose to abstract over 
$\syms{X}$'s $f$ dimension but not over the $f$ dimension of $\syms{Y}$, or 
vice-versa.  
This can prove useful if one of the symbolic sets is known to be a subset of 
the other, thus one would need to build an abstract filter only for the smaller 
one. 
Moreover, considering that initial or final transitions typically have the 
fewest number of matching events, one might choose to only collect and abstract 
over the symbolic sets of those transitions.
Due to their low cardinality these sets are likely to have very high 
filtering power and the data abstractions used to enforce them can achieve 
higher precision for the same operating costs (considering the small number of 
values to store and query).
Applying this heuristic to our example from section~\ref{sec:mot_example} wrt.\ 
to the final transition P (as we expect to have relatively few purchasing 
events) produces the following relaxed filters:
\begin{align*}
\heurs{S}(\symv{s}) 
&\equiv  
\slice{\syms{P}}
{(\facc{\symv{s}}{t}\,:\,\facc{\symv{s}}{t} + t_{out}),
	\facc{\symv{s}}{user}} 
\neq \emptyset
\\
\heurs{\oR}(\symv{r}) 
&\equiv
\slice{\syms{P}}{(\facc{\symv{r}}{t}\,:\,\infty), \facc{\symv{r}}{user}}
\neq \emptyset
\\
\heurs{P}(\symv{p}) 
&\equiv \texttt{true},
\end{align*}
as obtained 
from the definitions of $\precs{S}, \precs{\oR}$ and $\precs{P}$
by replacing $\syms{S}$ with the full domain of time/user ids and  
$\syms{\oR}$ with the empty set. 
By further applying data abstraction to these filters, we end up with a 
relatively cheap way to dispose of a large number of the input events, 
irrelevant to our pattern.  

\subsection{Building filters through fixpoint}


We remark that the abstract filters are not strictly tied to the  
expressibility of automata as relational queries, but that they can also be 
computed for an arbitrary automaton by using a fixpoint operator that keeps 
track of provenance information.
The fixpoint operator we consider assigns to the starting node the empty 
sequence and then iteratively builds for every node of the automaton its 
corresponding set of (partial) matches, until no more new matches are found. 
In every iteration, for each node $i$ and each of its outgoing transitions $X$, 
the partial matches added by the previous round to $i$ get extended if matching 
events are found within the symbolic set of $X$. 
The newly found matches then get added to the collection of matches 
corresponding to $X$'s destination, and the process starts over.  
Since we never remove partial matches we are guaranteed to reach a fixpoint.
Based on the complete matches collected by the accepting nodes, we also update 
the abstract filter's components corresponding to the transitions along their 
path.

In designing the fixpoint operator we similarly make extensive use of sets and 
set operators.
In particular, the (partial) matches we compute for each node $i$ of the 
automaton are represented as a sequence of sets, one for each transition that 
may occur on a path from the start node to $i$.
Then the abstract filter for a specific transition can be obtained by unioning 
its corresponding set across all the accepting nodes.
When deciding whether a partial match can be extended by triggering transition 
$X$, we simply intersect the symbolic set $\syms{X}$ with the projection from 
the partial match of the corresponding join fields, i.e.\ those fields of 
previously occurring transitions that are joined against $X$.
If the result is empty then the extension is not possible, otherwise a new 
partial match gets created whose corresponding set for transition $X$ is the 
result of the intersection.   





 

\section{Implementation}
\label{sec:implementation}

\begin{figure}[tp]
\centering
\includegraphics[clip, trim=5.6cm 4.5cm 6.3cm 4.2cm, 
width=0.45\textwidth]{graphs/query_plan.pdf}
\caption{Map-Reduce execution plans for pattern mining using: (i) the Baseline 
approach or (ii) Abstract Pattern Matching.}
\label{fig:query_plan}
\end{figure}


In the following we detail how we implemented our solution for speeding up 
large scale event series pattern matching on top of the 
Cosmos/Scope~\cite{Chaiken:2008} map-reduce framework.
We recall that the standard way of performing event-series pattern matching in 
such frameworks is to first remove all the events that do not match any of the 
selection predicates specified by the pattern, and then to sort on 
time/reshuffle the remaining events and finally process them using a pattern 
matching engine (see figure~\ref{fig:query_plan}(i)).

Our solution significantly extends the amount of data that gets filtered out 
during the preprocessing stage by constructing an {\em abstract} filter which 
also enforces the join predicates occurring in a pattern (as opposed to just 
the selection predicates).
The execution plan that we propose introduces 3 intermediary steps as is 
outlined in figure~\ref{fig:query_plan}(ii).
The first one collects in a parallel fashion a symbolic set for each event 
variable of the pattern (step 2a), followed by the union of these sets across 
all partitions of the input, and the results are then used to build the 
abstract filter by enforcing the join predicates between event variables (step 
2b). 
We then broadcast the abstract filter back to the preprocessing nodes and have 
them apply it over the output of the initial selection-predicate based filter 
(step 2c). 
Finally, the remaining events are sorted/reshuffled, just like in the standard 
approach and processed by the pattern matching engine. 


In implementing the execution plan in figure~\ref{fig:query_plan}(ii),
we make use as much as possible of the relational constructs and annotations 
provided by the Scope query language in order to maximize the potential for 
optimizations at the level of the entire data processing pipeline (for eg., we 
use native Scope to extract the event fields used in constructing the symbolic 
sets as well as to apply the filters that we build).
For operating with the set abstractions themselves we use the rich 
extensibility features of Scope, all the while providing hints to its query 
optimizer.

Even though the {\em abstract} filters have the potential to dramatically 
reduce the amount of data that gets sorted\allowbreak /\allowbreak 
reshuffled\allowbreak /\allowbreak pattern matched, 
constructing them introduces overheads both in terms of processing costs and 
latency.
In terms of processing costs we note that, while the construction of the 
symbolic sets can be done in the same pass as the selection-predicate based 
filtering, for applying the abstract filters we need a second pass over the 
data.
Nonetheless, since applying a filter is linear in the size of the input, this 
extra pass ends up being much cheaper than sorting on time the same input, and 
if the reduction ratio of the filter is considerable then applying it results 
in a net win. 
   
In terms of latency, our approach also introduces an extra reduction phase as 
required for aggregating the symbolic sets computed on each partition to a 
particular node where we can propagate the constraints specified by join 
predicates in order to obtain the abstract filters.
This is also followed by a broadcast phase that delivers the abstract filters 
back to the nodes processing the input stream.
In order to minimize the penalty in latency incurred by these two steps we 
limit the size of the abstract filters to the order of megabytes and we exploit 
the algebraic properties of the operators of the set abstractions that we use 
in order to optimize the aggregation (i.e.\ we make use of a {\em recursive} 
user defined aggregates).





  
  
  
  
\section{Evaluation}

\begin{figure}
\centering

\begin{subfigure}[t]{\columnwidth}
\includegraphics[clip, trim=0.8cm 1.5cm 0.9cm 2.15cm,
width=\columnwidth]{graphs/data_red.pdf}
\caption{Processed data reduction}
\label{fig:data_red}
\end{subfigure}
~
\begin{subfigure}[t]{\columnwidth}
\includegraphics[clip, trim=0.8cm 2cm 0.9cm 2cm,
width=\columnwidth]{graphs/processing_red.pdf}
\caption{Processing time speedup}
\label{fig:processing_red}
\end{subfigure}
~
\begin{subfigure}[t]{\columnwidth}
\includegraphics[clip, trim=0.8cm 2cm 0.9cm 2cm,
width=\columnwidth]{graphs/latency_red.pdf}
\caption{Latency speedup}
\label{fig:latency_red}
\end{subfigure}

\caption{Reduction ratios of processed data and processing speedups when using
abstract filters wrt.\ the corresponding baseline measures.}
\end{figure}



\begin{figure}[tp]
\centering

\begin{subfigure}[t]{\columnwidth}
\includegraphics[clip, trim=0.8cm 3.5cm 0.9cm 2.15cm,
width=\columnwidth]{graphs/A5_breakdown.pdf}
\caption{A5}
\label{fig:A5}
\end{subfigure}
~
\begin{subfigure}[t]{\columnwidth}
\includegraphics[clip, trim=0.8cm 4cm 0.9cm 4cm,
width=\columnwidth]{graphs/A3_breakdown.pdf}
\caption{A3}
\label{fig:A3}
\end{subfigure}

\caption{Breakdown of the total processing time (in hours) for queries A5 and
A3.}
\label{fig:breakdown}
\end{figure}


\begin{figure}[tp]
\centering

\includegraphics[clip, trim=0cm 3.8cm 0.3cm 10cm,
width=\columnwidth]{graphs/A5_tradeoff.pdf}
\caption{Tradeoff for query A5 between filter size (1024KB-8192KB) and data
reduction for different filter configurations and with different granularities
for the MainKey filter (65K,512K,2M).}
\label{fig:tradeoff}
\end{figure}




We tested our approach on 2 workloads:
i) Asimov, consisting of 15 patterns (A1-15) matched over telemetry events
collected within a 2 hours interval by the Asimov system, and 
ii) GitHub, performing repository analytics (patterns G1-3) over the GitHub
dataset consisting of events collected over a 5 years period.
All transitions in a pattern have a main key constraint, i.e.\ all events in a
match should belong to the same device for the Asimov workload, or the same
repository for the GitHub patterns, and a time constraint, i.e.\ all events 
considered should occur within a timeout from the initial event in the match.
In addition, some patterns also have a secondary key constraint between some of
their transitions.

We assess the benefits provided by our approach in terms of 
the ratio of input events vs selected events, 
the total execution time across the processing nodes of the cluster,
and the time it takes to complete the query (latency).
We also use query A5 to highlight the individual filtering potential
of different join predicates, as well as their sensitivity wrt.\ the amount of
state dedicated to their symbolic representation.



\subsection{Processed data reduction}

In order to establish the raw potential of our approach we look at the amount of
data that is discarded by the abstract filters that we build.
We note that the baseline consists only of events whose type matches the event
type of a transition in the query.

Figure~\ref{fig:data_red} shows on a logarithmic scale the ratio between the
data processed by the baseline approach and the data processed by our solution
when considering different join predicates for building the abstract filters.
In particular, we look at three types of join predicates: those referencing
the main join key of the query (MainKey), those imposing constraints on the
timestamp of matching events (Time), and those referencing a secondary join key
(SecondaryKey).
We report the results provided by the MainKey filter as well as its combination
with the Time and SecondaryKey filter (for queries with joins on a secondary
key).
 
For 12 out of the 18 queries in our workload we obtain at least a 10x reduction
in the amount of data that has to be considered by the pattern matcher, with 3
of them ending up processing 5 orders of magnitude less data.
The other 6 queries exhibit modest or no data reduction. This is mainly due to
the precision lost by our abstract filters, as well as the fact that not all
join predicates from a query can be efficiently abstracted over. 

While incorporating extra join predicates in the abstract filters leads to
further savings, it is dependent on the query and the stream of events which
additional join predicate would provide the most benefits.
In our workload we observe that adding the Time filter for queries A5 and A8 
provides the biggest improvement while for queries A2-4, A10, A14 and G1 the
SecondaryKey filter is more advantageous. 
Due to this variability one has to decide on a query by query basis which join
predicates to use and how much state to allocate for them within the abstract
filter.
For our workload we assign to the MainKey, Time and SecondaryKey filters between
16KB and 4MB, 8B and 180B, and 2B and 8B, respectively.
This allocation has been chosen under the constraint of a total size for the
abstract filter in the order of megabytes and while considering the
particularities of queries, for example, we take into account the fact that a
query with a large timeout window would not benefit from a finegrained Time
filter.




\subsection{Abstract filter size vs filtering ratio}

We vary the size of the abstract filter that we build as well as its
configuration in order to asses the impact on the reduction ratio it provides.
In particular, we experimented for query A5 with filters of sizes between 1024
to 8192KB and with a granularity for the MainKey filter between 65K to 2M
distinct values. For each abstract filter size and MainKey granularity we devote
the rest of available space to either the Time or the SecondaryKey filter.
Since the filters we build are multiplicative in their composition, the
granularity afforded to the Time and SecondaryKey filters varies between 4 and
1024 distinct values depending on the granularity of the MainKey filter as well
as the total size of the abstract filter.
In addition we record for reference the reduction of just the MainKey filter for
its different configurations.
From the results presented 



\subsection{Total processing time speedup}

In the following we look at the total processing cost across all nodes in the
cluster.
This is a particularly relevant metric for cluster setups that allow workload
consolidation or data centers that charge users based on total number of
``processing hours'' consumed.
Figure~\ref{fig:processing_red} shows speedups of 1.25x to 2.14x for 11 out of
the 18 queries, while 5 patterns experience slowdowns of at most 0.62x.
The slowdowns are a consequence of performing an additional pass over the input
in order to build the abstract filters, with little or no benefit in terms of
discarded events.

In order to highlight the tradeoffs of our approach we break down the {\em
baseline} execution of a query into the time it takes to 
i) read and sort the data, and 
ii) perform the reduction step.
On the other hand for our technique we look at the time it takes to 
i) read the data, 
ii) build the symbolic state associated with each transition,
iii) reduce the symbolic state associated to each transition down to an abstract
filter for the entire query and broadcast it back to every mapper,
iv) filter the input events based on the abstract filter, and
v) perform the reduction step over the remaining events. 

By discarding some of the input events our solution cuts the cost
of the sorting and reduction phases when compared to the baseline approach,
while adding the overhead of building and applying the abstract filters.
When the number of events removed is significant this leads to an overall lower
processing time as can be seen in figure~\ref{fig:breakdown}.
Since every join predicate included in the abstract filter incurs additional
costs when building and applying the filter, to maximize performance one should
consider only the smallest/cheapest set of predicates able to filter the input
down to the order of gigabytes/tens of gigabytes.


In particular we remark that the additional reduction in data provided by the
Time filter for query A5 does not translate in lower processing time (see
figure~\ref{fig:A5}).
This happens because the default MainKey filter already drastically reduces
the amount of data that needs to be processed by the pattern matcher. 
Therefore the performance gain from further reduction is easily canceled by the
cost of building the additional filter.
A similar situation occurs in query A3 where the sweet-spot is provided by the
combination of MainKey and Time predicates, even though the combination of
Main and Secondary Key filters manages to discard the largest number of events 
(figure~\ref{fig:A3}).


\subsection{Latency speedup}


In terms of end-to-end running times we record speedups between 1.08x and 1.78x
for 8 out of the 18 queries in our workload, while for 4 of them our
approach performs within 5\% of the baseline (see Figure~\ref{fig:latency_red}).
Just like in the case of processing times, the slowdowns in latency mainly
correspond to queries for whom the abstract filters do not significantly reduce
the amount of data processed by the pattern matcher (below 5x).


We examine the performance of queries A12 and A13 as it highlights another
important factor in the latency speedup achieved by our approach.
While for both queries the abstract filters discard a significant number of the
input events and as such have smaller total processing times than the baseline,
only for A12 this translates in smaller end-to-end running times.
This happens due to the difference in the initial size of the input, 1.2TB for
A12 vs 307GB for A13, which when processed on a 85 nodes cluster results in
average running times of the reduction phase of 3.8 minutes for A12 vs 46
seconds for A13. 
Even though in our approach the reduction phase for A13 takes only 5.6 seconds
on average due to the smaller number of events being processed, that is not
enough to compensate for the additional latency introduced by the phases that
build and broadcast the abstract filters.
While the latency of these phases can be minimized by decentralizing the process
of building the abstract filters, developing and deploying such techniques falls 
outside the scope of our current work.
% same reason for A15


















\section{Conclusions and Future work}
\label{sec:conclusions}



\small
\bibliographystyle{abbrv}
\bibliography{main}




\end{document}
