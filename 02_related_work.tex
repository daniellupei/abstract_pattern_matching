\section{Related Work}
\label{sec:rel_work}

Complex event processing has received extensive interest in the 
literature~\cite{SASE:2008,SASE:2014,
	Brenna:2007,Demers:2007,Brenna:2009,	
	Cadonna:2011} and 
has enjoyed similarly wide adoption in 
industry~\cite{esper_epl,oracle_mr,aster_npath,Carasso:2012,flink_cep},
with most of the work focused on online (near-real time) systems designed to 
deliver extremely low response times for mission-critical tasks (for eg., 
blocking a credit card upon detecting fraudulent activity). 
However, 
these systems experience scalability issues 
as the volume of data that needs to be processed continues to grow
and considering that the pattern matching logic is typically implemented in a 
sequential manner.  
Moreover, the optimizations that they deploy focus primarily on computation 
reuse and sharing~\cite{Sadri:2004}, and thus are not appropriate for 
addressing the demand for more parallelism.

In dealing with the increase in load, most systems dismiss, as early as 
possible, 
of all the input events that fail to satisfy the selection predicates 
of at least one of the variables in a given pattern.
A more aggressive \textbf{preprocessing} step has been proposed by Cadonna et.\ 
al.\cite{Cadonna:2012}, which leverages both the structure of the pattern and 
the guards of its variables to formulate a set of necessary conditions for a 
window of events to contain a complete match.
If a particular window fails to meet those conditions its can be ignored, thus 
completely sidestepping the pattern matcher.
While their proposal targets only patterns specified as {\em sequenced event 
sets}~\cite{Cadonna:2011}, we support arbitrary automata, and for a large 
class we derive both necessary and sufficient (i.e.\ precise) conditions for 
identifying complete matches.
More importantly, we discuss how the precision of these filters can be traded 
in favor of low overheads by proposing {\em data and predicate} abstractions, 
and we showcase their ability to minimize data shuffling  
when performing pattern matching over a map-reduce framework.    
 

%(*) Distributed pattern matching
Solutions for \textbf{distributed/parallel pattern matching} have focused 
mainly on three directions:
a)
partitioning the input, either by time windows~\cite{Leghari:2015} or key 
attributes~\cite{Hirzel:2012}, and then processing each partition  
sequentially,
b) 
partitioning the pattern via query plans~\cite{Akdere:2008} that produce the 
output from 
progressively larger sub-patterns and 
where the intermediary results are mined on distinct processing nodes,
and 
c)
partitioning the set of partial matches/runs that need to be explored at any 
given time~\cite{Balkesen:2013}.  
The first approach is vulnerable to data skew, in case of large time windows or 
attribute keys with few distinct values, 
while the second one does not support the Kleene star.
More importantly, all of them target only the pattern matching process 
itself and disregard the communication costs incurred when the analysis is 
performed on a map-reduce platform.
By contrast,
our technique of {\em abstract pattern matching}
limits the number of events that need to be considered by the pattern 
matching operator in the first place, and as such can even be used in 
conjunction with the proposals described above.
Moreover,
in our work we expose the inherent parallelism of patterns by expressing them 
in terms of embarrassingly parallel relational operators.  

Symbolic execution~\cite{King:1976} has been proposed as a way to speedup 
 user-defined aggregates~\cite{Raychev:2015}, and in particular 
pattern matching routines executing within the reduce stages of map-reduce 
workloads.    
The user defined aggregate is symbolically evaluated on each 
partition of the input and only the symbolic summary of the 
execution is submitted  to the reducer. 
The reducer can then determine the final result based on the collected symbolic 
summaries. 
This approach ends up shuffling in many cases significantly less data between 
mappers and reducers as the symbolic summaries are typically much smaller than 
the original input.
However, as it requires that the data be already ordered by time, it is not 
applicable to the common scenario where workload wide considerations impose a 
different physical layout for the input data.  
Moreover, even if the ordering constraint on the input holds, this approach can 
still leverage {\em abstract pattern matching} to address
the pathological scenarios when the symbolic summaries are just as large as the 
original input, as it would allow it to more accurately prune unfeasible paths 
based on a global view of the domain of join attributes. 



Proposals for coping with {\bf out-of-order event series} have 
primarily focused on buffering mechanisms and
 efficiently updating the internal structures of the pattern matcher in 
 response to events arriving late~\cite{Johnson:2007,Li:2007}.
Since they operate under the assumption that misplaced events are relatively 
rare (i.e.\ the input stream is mostly ordered),
use time thresholds beyond which delayed events are simply ignored,
and rely on a complete view of the input, they cannot mitigate the expensive 
data shuffling that precedes pattern matching on map-reduce platforms.   
 


%(*) Multiway join / semi-join optimizations


