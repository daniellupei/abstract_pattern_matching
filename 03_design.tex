\section{Design}

We consider queries specified as a symbolic finite automaton where each
transition is annotated by a variable and a guard.
The variable is used to bind the fields of the currently considered event while the guard is a propositional formula in disjunctive normal form which decides whether the transition can be triggered or not.  
The atomic formulas of the guard are either {\em selection} predicates, which only reference the variable associated with the current transition,
or {\em join} predicates, which may also reference the variables of preceding transitions.
In particular, the guards of start transitions can only use selection predicates.  
   
Pattern matching algorithms require the input stream of events to be (partially) sorted on time, while this is not always the case in practice due to competing constraints of other stages in the data processing pipeline. 
Thus, evaluating pattern matching queries in a map-reduce framework usually adds a reduction step in order to sort the input, which can become the main bottleneck of the analysis, both at the network level (large amounts of shuffled data) and at
the processing level (in case of data-skew).

 

We propose to significantly reduce the amount of data fed into the pattern
matcher (reducer) by dropping in a map phase all the events that cannot take
part in a successful match.
To do so we design a set of filters that while conservative, closely match the
semantics of the patterns analyzed.

\begin{comment}
We propose three levels of abstraction.
The first enforces the join constraints between different transitions as
expressed by join predicates within the transition guards.
The second one further imposes time windowing constraints (all events of a
successful match must occur within a timeout of the first event in the match).
Finally the last one enforces ordering constraints between {\em consecutive}
transitions of the pattern.
\end{comment}

In our approach we leverage the fact that a large class of symbolic finite
automata fit within the fragment of star-free languages which
have a corresponding FOL formula (possibly with counting quantifiers).
Whenever that is not the case we can narrow the scope of our filters to the
sections of the pattern that do.
 
Based on these FOL formulas we produce precise filters, one per transition, 
that can be evaluated independently on each mapper.
Finally we further coarsen these filters to a set of relaxed filters
that can be collected and tested in a time and space efficient manner (for
example using bloom filters, time-interval maps).

We explore the trade-offs between the overheads incurred in building/querying
the filters and their accuracy. 


\subsection{Translating patterns into queries}

We assume as input an automata where each transition $X$ is annotated by a
predicate $p_X$ that may refer to the fields of the currently considered event,
as well as the events matched by the preceding transitions within the current
run.
To streamline the presentation we begin by considering automatas with only
cycles of length 1 and we distinguish between cycle transitions and
non-cycle transitions.


The translation process produces one query $Q_X$ per non-cycle transition $X$
and one query $Q_i$ per node $i$, excluding the initial node.
The transition queries $Q_X$ compute partial matches corresponding to the paths
ending with transition $X$.
Analogously, the node queries $Q_i$ produce partial matches corresponding to the
paths between the starting node and node $i$.
If a node $i$ is an accepting node then its query will return complete matches,
thus the final query of the automata is obtained by unioning all the queries
generated for the automata's accepting nodes.


The schema of these queries consists of a group of fields for each non-cycle
transition that may occur along its associated set of paths. If a transition is
triggered within a partial match then its corresponding group of fields records
the timestamp of the event that triggered it along with any other event
attributes that may be referenced by the predicates of succeeding transitions.
On the other hand, if a transition is not triggered in a particular partial
match then its group of fields is left void.


The translation process iterates in topological order over the nodes of the DAG
obtained by ignoring the cycle transitions of the automata.
At each node $i$, it first generates the queries for all its incoming
non-cycle transitions $Q_X$ and then $Q_i$ is defined as their union. 
The schema of $Q_i$ is established as the
union of the schemas of the incoming transitions $Q_X$, and the necessary tuple
mapping from the schema of $Q_X$ to the schema of $Q_i$ is also undergone.

The query for non-cycle transition $X$ with source node $k$ is generated as a
join between the partial matches produced by query $Q_k$ and the input
relation of events.
The possible event matches $e_X$ for transition $X$ are constrained based on the
transition's predicate $p_X$, and $e_X$'s timestamp which should be greater than
that of the latest event $e_k$ from the partial match computed by $Q_k$.
In addition, an inner query specifies that no other event should occur between
$e_k$ and $e_X$, apart from those matching a cycle transition on node $k$, if
such a transition should exist.



\subsection{Precise filter generation}
\label{sec:prec_filter_generation}

Based on the automata query $Q$ derived in the previous section we can generate
a set of precise filters, one per transition, such that if an event does not match
any of them, it is guaranteed not to contribute to any complete match.

The precise filter corresponding to a non-cycle transition $X$ is obtained from
$Q$ by projecting away (i.e.\ existentially quantifying) all the other transitions
from the complete match returned by $Q$.

For cycle transitions $Y$ on a node $k$ we define its precise filter as a
conjunction between 
a) the associated predicate $p_Y$ while existentially quantifying its join
attributes over the fields stored in the complete matches returned by $Q$, and
b) the time constraint specifying they should only occur between events matching
an incoming and an outgoing transition of $k$ within a complete match produced
by $Q$.

\subsection{Symbolic filters via data abstraction}
\label{sec:data_abstraction}


\subsection{Predicate abstraction}
\label{sec:pred_abstraction}


We propose {\em predicate abstraction}, ie.\ the technique of strengthening the
precise filters by discarding some of their predicates, as a way of overcoming
the challenges that can arise when turning precise filters into symbolic filters
via data abstraction.
For example, it may happen that for some predicate types
(for eg.\ $x.Contains(y)$, where $x$, $y$ are strings) we simply cannot provide
any data abstraction, and even for those that we can, materializing and querying
those data abstractions might prove too expensive.

Predicate abstraction is an essential component of our approach
allowing us to strike the right balance between the data reduction that the
symbolic filters provide on one hand, and the overheads introduced by their
data abstractions on the other.
For instance, we may choose to discard
predicates that have very low selectivity, i.e.\ the reduction in input data
that they provide does not justify the cost of enforcing them.
Similarly, one may turn to predicate abstraction when dealing with patterns with
a large number of join predicates or transitions, in order to mitigate the
increased overheads incurred by their data abstractions.

Considering that initial or final transitions typically have the fewest
number of matching events, one might choose to generate filters that only
consider the join predicates with respect to those transitions.
Due to the low cardinality of the sets of events matching these transitions 
the join predicates involving them are likely to have very high selectivity.
Moreover, the data abstractions used to enforce them can achieve higher
precision for the same operating costs.












 
